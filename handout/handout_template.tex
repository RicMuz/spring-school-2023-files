%TEMPLATE
%for Spring School 2023
%Don't change any macros given.
%The first place for your change is below line 250.
%See you at Spring School
%%%%%%%%%%%%%%%%%%%%%%%%%%%%%%%%%%%%%%%%%%%%%%%%%%%%%%%%%%%%%%%
%DO NOT CHANGE THIS
%
\documentclass[12pt]{report}
%%%%%%%%%%%%%%%%%%%%%%%%%%%%%%%%%%%%%%%%%%%
%PACKAGES
%basic
\usepackage[utf8x]{inputenc}
\usepackage[T1]{fontenc}
\usepackage{lmodern}
\usepackage[english]{babel}
\usepackage[unicode]{hyperref}%%sazba url
%apperance
%\usepackage{fullpage}
\usepackage[margin=0.75in]{geometry}
\usepackage[hyphenbreaks]{breakurl}
\usepackage{multicol}  %% for multiple columns in text
%schedule
\usepackage[usenames,dvipsnames]{color}
%math and usefull packages
\usepackage{amsfonts}
\usepackage{amsmath, amssymb}
\usepackage{graphicx}
\usepackage[linesnumbered, vlined]{algorithm2e}%algorithms
% tikz
\usepackage{tikz}


%%%% by participants:
%%%%%%%%%%%%%%%%%%%%%%%%%%%%%%%%%%%%%%%%%%%%%%%%%%%%%
%MACROS
%%%%%%%%%%%%%%%%%%%%%%%%%%%%%%%%%%%%%%%%%%%%%%%%%%%
%TITLE
\makeatletter

\def\pen{\penalty10000}


\newdimen\HeaderBoxMargin
\HeaderBoxMargin=3pt

\def\HeaderBox#1{
  \hbox{%
    \vrule width 0.4pt \hglue -0.4pt%
    \vbox{%
      \hrule height 0.4pt\pen\vskip0.5mm\hrule height 0.4pt\pen\smallskip%
      \begin{center}%
        % velmi velmi ošklivý hack, aby se text nedotýkal stran
        \leftskip=\the\HeaderBoxMargin plus 1fil%
        \rightskip=\the\HeaderBoxMargin plus 1fil%
        #1%
      \end{center}%
      \medskip\hrule height 0.4pt%
    }%
    \hglue -0.4pt \vrule width 0.4pt%
  }%
  \pen\smallskip%
}


\def\chapter*#1{\Heading{#1}}%


\long\def\literaturestart#1\literatureend{{
  \def\chapter*##1{\section{##1}\vspace*{-\parskip}}
  \begin{thebibliography}{9}
  \footnotesize\normalfont
    #1
  \end{thebibliography}
}}


% sázení section v obsahu vlevo a ne odsazeně
\def\l@section{\@dottedtocline{1}{0em}{2.3em}}

% okouše mezery z konce argumentu (protože unskip nefunguje v pdf obsahu)
\def\Trim@Impl#1#2 \end#3\Trim@End{\ifx\end#3\end#1\else\Trim{#2}\fi}
\def\Trim#1{\Trim@Impl{#1}#1\end{} \end\Trim@End}


\newcommand{\Heading}[1]{
\begin{center}
\bf \LARGE #1
\end{center}
\bigskip
\bigskip

}

%%%%%%%%%%%%%%%%%%%%%%%%%%%%%%%%%%%%%%%%%%%%%%%%%%%
%TITLE type

% jen obalí argumenty \Trim{} a pokud je #1 prázdné
% tak ho nastaví na \Trim{#2}\Trim{#3}
\long\def\Id#1\FirstName#2\Surname#3\Email#4\Serie#5\Title#6%
         \PaperAuthor#7\Link#8\Text#9\End{{%
  \IdImpl\ifx\XXX#1\XXX\Trim{#2}\Trim{#3}\else\Trim{#1}\fi
    \FirstName\Trim{#2}\Surname\Trim{#3}\Email\Trim{#4}%
    \Serie\Trim{#5}\Title\Trim{#6}%
    \PaperAuthor\Trim{#7}\Link\Trim{#8}\Text #9\End%
  }}

\def\Id@title#1{\medskip\\{\Large #1}}

\long\def\IdImpl#1\FirstName#2\Surname#3\Email#4\Serie#5\Title#6%
         \PaperAuthor#7\Link#8\Text#9\End{{%
  \def\nop{\Trim{}}%
  \def\testemail{#4}%
  \def\testserie{#5}%
  \def\testpaper{#7}%
  \def\testlink{#8}%
  %
  \phantomsection%
  \HeaderBox{%
    {\Large #2 #3}\pen%
    \ifx\testemail\nop\else%
      \smallskip\\{\tt #4}\pen%
    \fi%
    \ifx\testpaper\nop\else%
      \smallskip\\{Presented paper by #7}\pen%
    \fi%
    \ifx\testserie\nop%
      \Id@title{#6}%
      \addcontentsline{toc}{section}{#6 (\nobreak{#2} \nobreak{#3})}%
    \else%
      \Id@title{#6\penalty-2{} {\it as part of series} #5}%
      \addcontentsline{toc}{section}{#5: #6 (\nobreak{#2} \nobreak{#3})}%
    \fi%
    \ifx\testlink\nop\else%
      \smallskip\\{(\texttt{#8})}\pen%
    \fi%
    \label{#1}%
  }%
  \setcounter{theorem}{0}%
  {#9}%
}}

%name surname email {Title}
\newcommand{\Talk}[4]{%
  \Id\FirstName #1 \Surname #2 \Email #3
  \Serie
  \Title #4
  \PaperAuthor
  \Link
  \Text\End
}
%name surname email {author of the paper}{Title}{link}
\newcommand{\PaperTalk}[6]{%
  \Id\FirstName #1 \Surname #2 \Email #3
  \Serie
  \Title #5
  \PaperAuthor #4
  \Link #6
  \Text\End
}
%name surname email {author of the paper}{Title}{Serie}{link}
\newcommand{\SeriesTalk}[7]{%
  \Id\FirstName #1 \Surname #2 \Email #3
  \Serie #6
  \Title #5
  \PaperAuthor #4
  \Link #7
  \Text\End
}
%name surname email {Title}{Serie}
\newcommand{\SeriesOther}[5]{%
  \Id\FirstName #1 \Surname #2 \Email #3
  \Serie #5
  \Title #4
  \PaperAuthor
  \Link
  \Text\End
}
%%%%%%%%%%%%%%%%%%%%%%%%%%%%%%%%%%%%%%%%%%%%%%%%%%%
%Schedule macros
%Schaduled Talk: Talk Speaker_Name Speaker_Surname Time Duration 
\def\ST#1#2#3#4#5{\Event{#5}{#1}{#2 #3 \hskip 0pt plus 0.5fil \mbox{#4} \hskip 0pt plus 0.5fil \pageref{#2#3}}}
%%%%%%%%%%%%%%%%%%%%%%%%%%%%%%%%%%%%%%%%%%%%%%%%%%%
% Math enviroment definition.
\newcounter{theorem}
\newtheorem{definition}[theorem]{Definition}
\newtheorem{theorem}{Theorem}
\newtheorem{lemma}[theorem]{Lemma}
\newtheorem{observation}[theorem]{Observation}
\newtheorem{corollary}[theorem]{Corollary}
\newtheorem{conjecture}[theorem]{Conjecture}
\newenvironment{proof}{\par\textbf{Proof}}{ \rule{1pt}{0pt}   \hfill $\square$ \\}
\newtheorem{proposition}[theorem]{Proposition}
\newtheorem{note}[theorem]{Note}

\newtheorem{fact}[theorem]{Fact}
\newtheorem{exercise}[theorem]{Exercise}

%%%%%%%%%%%%%%%%%%%%%%%%%%%%%%%%%%%%%%%%%%%%%%%%%%%
% Dela sekce v jednotlivych kouskach sbornicku
\def\subsection#1{\par Not good idea. \smallskip}
\def\section#1{\vskip 9pt plus 2pt minus 1pt\centerline{\bf #1}\par\ignorespaces}
\def\paragraph#1{\par{\bf #1}\hskip 0.5em\ignorespaces}
%%%%%%%%%%%%%%%%%%%%%%%%%%%%%%%%%%%%%%%%%%%%%%%%%%%
%% Pro formatovani odstavcu
\setlength{\parindent}{0pt}
\setlength{\parskip}{5pt plus 1pt minus 1pt} % smallskip  size (was 0.5em)
\setlength{\parsep}{\parskip}
\def\sirka{0.89\textwidth}

\let\trivlist@old\trivlist
%\def\trivlist{\parskip=0pt\partopsep=0pt\trivlist@old}

\let\normalsize@old\normalsize
\def\normalsize{%
  \normalsize@old%
  \setlength{\abovedisplayskip}{5pt plus 3pt minus 2pt}% def 10 +2 -5 pt
  \setlength{\belowdisplayskip}{5pt plus 3pt minus 2pt}% def 10 +2 -5 pt
  \setlength{\abovedisplayshortskip}{0pt plus 3pt}% def 0 +3 pt
  \setlength{\belowdisplayshortskip}{0pt plus 3pt}% def 6 +3 -3 pt
}

\setlength{\itemsep}{0pt} % def 4.0pt plus 2.0pt minus 1.0pt
\setlength{\topsep}{0pt}
\setlength{\partopsep}{0pt}

%% Makro pro vkladani prispevku
\newcommand{\prispevek}[1]{
	\input{prispevky/#1}\vskip 3em
}
%%%%%%%%%%%%%%%%%%%%%%%%%%%%%%%%%%%%%%%%%%%%%%%%%%%
%% Math macros
\newcommand{\T}[1]{#1^\top\!} 
\newcommand{\calO}{{\mathcal O}}

\newcommand{\NP}{\ensuremath{\mathrm{NP}}}
\newcommand{\FPT}{\ensuremath{\mathrm{FPT}}}
\newcommand{\Pp}{\ensuremath{\mathrm{P}}}

\def\epsilon{\varepsilon}
\def\phi{\varphi}

\pdfstringdefDisableCommands{%
  \def\chi{\textchi}%
  \def\Delta{\textDelta}%
  \def\omega{\textomega}%
}

\makeatother

%%%%%%%%%%%%%%%%%%%%%%%%%%%%%%%%%%%%%%%%%%%%%%%%%%%%%%%%%%%%%%%%%%%%
%END OF DO NOT CHANGE THIS
%%%%%%%%%%%%%%%%%%%%%%%%%%%%%%%%%%%%%%%%%%%%%%%%%%%%%
%
%
%
%%%%%%%%%%%%%%%%%%%%%%%%%%%%%%%%%%%%%%%%%%%%%%%%%%%%%%%%%%%%%%%
%Your macros
%Please use as few as possible.
%\newcommand{\mbb}[1]{\ensuremath{\mathbb{#1}}}
%
%
%
%
%%%%%%%%%%%%%%%%%%%%%%%%%%%%%%%%%%%%%%%%%%%%%%%%%%%%%%%%%%%%%%%
%Your packages
%Please use as few as possible.
% \usepackage{yourfavouritepackage}
%
%
%
%
%%%%%%%%%%%%%%%%%%%%%%%%%%%%%%%%%%%%%%%%%%%%%%%%%%%%%
%DOCUMENT
\begin{document}
%%%%%%%%%%%%%%%%%%%%%%%%%%%%%%%%%%%%%%%%%%%%%%%%%%%%%%%%%%%%%%%
%Please fill the informations
% Choose only one of the following macros and comment out the rest.
% If you will present someones paper
%\PaperTalk{Your Name}{Surname}{your@email.cz}{Author of the article}{The Title}{http://link.springer.com/article/10.1007\%2FBF02579200}
%If you present a paper as a part of series.
%\SeriesTalk{Your Name}{Surname}{your@email.cz}{Author of the article}{The Title}{The Best Series}{http://link.springer.com/article/10.1007\%2FBF02579200}
%Other talk as a part of series.
\SeriesOther{Richard}{Mužík}{richard@imuzik.cz}{Introduction to Cooperative Game Theory}{Cooperative game theory}
%Some other talk
%\Talk{Richard}{Mužík}{richard@imuzik.cz}{The Introduction to Cooperative Game Theory}
%%%%%%%%%%%%%%%%%%%%%%%%%%%%%%%%%%%%%%%%%%%%%%%%%%%%%%%%%%%%%%%
%The beginning of your write up.
\section{Introduction}
In this lecute you will find brief introduction to cooperative games.

\begin{definition}[Cooperative game]
  A cooperative game is an ordered pair $(N,v)$, where $N$ is a set of players and $v\colon 2^N \to \mathbb{R}$ is the characteristic function. Further, $v(\emptyset) = 0$.
\end{definition}

\begin{definition}[Payoff vector]
  Payoff vector is $x \in \mathbb{R}^n$, where $x_i$ represents payoff of player $i$. It is efficient, if $\sum_{i \in N} x_i = v(n)$. It is individually rational, if $x_i \geq v(i)$.
\end{definition}

\begin{definition}[Core]
  For a cooperative game $(N,v)$, the core $\mathcal{C}(v)$ is
  \[
    \mathcal{C}(v) = \left\{x \in \mathcal{I}^*(v) \mid x(S) \geq v(S), \forall S \subseteq N \right\}.
  \]
\end{definition}

\begin{observation}[Emptyness of the core]
  There are cooperative games $(N,v)$ with empty core.
\end{observation}

\begin{theorem}[Weak Bondareva-Shapley]
  Cooperative game $\left(N,v\right)$ has non-empty core if and only if 
	\[
		v\left(N\right) \geq \sum_{S \subseteq N} y_S v\left(S\right)\text{ for all feasible }y \in \mathbb{R}^{\left(2^n-1\right)}
	\]
\end{theorem}

\begin{definition}
  For a cooperative game $(N,v)$, the Shapley value $\phi(v)$ of player $i$ is
  \[
    \phi_i(v) = \sum_{S \subseteq N \setminus i}\frac{s!(n-s-1)!}{n!}\left(v(S \cup i) - v(S)\right)
  \]
\end{definition}

\begin{definition}[Marginal vector]
  For cooperative game $(N,v)$ and permutation $\sigma \in \Sigma_n$ is $m_v^\sigma$ marginal vector, where $\left(m_v^\sigma\right)_i = v\left(S_{\sigma(i)}\cup i\right)-v\left(S_{\sigma(i)}\right)$ and $S_{\sigma(i)}= \left\{j \in N \mid \sigma(j) < \sigma(i)\right\}$
\end{definition}

\begin{definition}[Weber set]
  For cooperative game $(N,v)$ the Weber set is
  \[
    \mathcal{W}(v)=conv\left\{m^{\sigma}_{v}\mid \sigma \in \Sigma_n\right\}
  \]
\end{definition}

\begin{lemma}[Shapley value and Weber set]
  For a cooperative game $(N,v)$ it holds:
	\[
    \phi(v) \in \mathcal{W}(v)
  \]
	Moreover, $\phi(v)$ is the center of gravity of $\mathcal{W}(v)$.
\end{lemma}

\begin{theorem}[Weber set and core]
  For every cooperative game $(N,v)$, it holds $\mathcal{C}(v) \subseteq \mathcal{W}(v)$
\end{theorem}

\begin{definition}[Classes]
  The cooperative game $(N,v)$ is said to be
  \begin{itemize}
    \item monotonic game $\iff$ $\left(S \subseteq T \subseteq N\right)\left(v(S) \leq v(T)\right)$
		\item superadditive game $\iff$ $\left(S,T \subseteq N, S \cap T = \emptyset\right)\left(v(S)+v(T) \leq v\left(S \cup T\right)\right)$
		\item convex game $\iff$ $\left(S,T \subseteq N\right)\left(v(S)+v(T) \leq v\left(S \cap T\right)+v\left(S \cup T\right)\right)$
		\item essential game $\iff$ $v(N) \geq \sum_{i \in N} v(i)$
		\item balanced game $\iff$ $\mathcal{C}(v) \neq \emptyset$
  \end{itemize}
\end{definition}

\begin{theorem}[Balanced and essential]
  Balanced cooperative games are essential.
\end{theorem}

\begin{theorem}[Convex and superadditive]
  Convex cooperative games are superadditive.
\end{theorem}

\begin{theorem}[Core of convex games]
  For a convex cooperative game $(N,v)$, it holds $\mathcal{C}(v)=\mathcal{W}(v)$
\end{theorem}

\begin{corollary}[Shapley value and convex games]
  For a convex cooperative game $(N,v)$, it holds:
	\begin{enumerate}
		\item $\phi(v) \in \mathcal{C}(v)$
		\item $\phi(v)$ is the centre of gravity of $\mathcal{C}(v)$
	\end{enumerate}
\end{corollary}

%%%%%%%%%%%%%%%%%%%%%%%%%%%%%%%%%%%%%%%%%%%%%%%%%%%%%%%%%%%%%%%
%In case you would like to have some refences.
%It is ussualy not neccessary.
%Please, uncomment/comment start and end of the bibliography
\literaturestart
	
%please add a literature in given format as in examples:
\bibitem{latexcompanion} 
Hans Peters 
\textit{Game theory: A multi-leveled approach}. 
Springer Texts in Business and Economics, 2nd edition, 2015.
 

\literatureend
%%%%%%%%%%%%%%%%%%%%%%%%%%%%%%%%%%%%%%%%%%%%%%%%%%%%%%%%%%%%%%%
%end of the document
%PLEASE DO NOT CHANGE
\end{document}
