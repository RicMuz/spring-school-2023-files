%%%%%%%%%%%%%%%%%%%%%%%%%%%%%%%%%%%%%%%%%
% Focus Beamer Presentation
% LaTeX Template
% Version 1.0 (8/8/18)
%
% This template has been downloaded from:
% http://www.LaTeXTemplates.com
%
% Original author:
% Pasquale Africa (https://github.com/elauksap/focus-beamertheme) with modifications by 
% Vel (vel@LaTeXTemplates.com)
%
% Template license:
% GNU GPL v3.0 License
%
% Important note:
% The bibliography/references need to be compiled with bibtex.
%
%%%%%%%%%%%%%%%%%%%%%%%%%%%%%%%%%%%%%%%%%

%----------------------------------------------------------------------------------------
%	PACKAGES AND OTHER DOCUMENT CONFIGURATIONS
%----------------------------------------------------------------------------------------
\documentclass{beamer}
%\documentclass[handout]{beamer}

\usepackage{mathtools}
\usepackage{comment}

\usetheme{focus} % Use the Focus theme supplied with the template
% Add option [numbering=none] to disable the footer progress bar
% Add option [numbering=fullbar] to show the footer progress bar as always full with a slide count

% Uncomment to enable the ice-blue theme
%\definecolor{main}{RGB}{92, 138, 168}
%\definecolor{background}{RGB}{240, 247, 255}

\definecolor{mygreen}{RGB}{0, 128, 0}
\definecolor{myblue}{RGB}{51, 51, 204}
\definecolor{myviolet}{RGB}{255, 0, 255}

\newcommand{\red}[1]{\textcolor{red}{#1}}
\newcommand{\green}[1]{\textcolor{mygreen}{#1}} 
\newcommand{\blue}[1]{\textcolor{myblue}{#1}}
\newcommand{\violet}[1]{\textcolor{myviolet}{#1}}

\newcommand{\R}{\mathbb{R}}
\newcommand{\Rn}[1][n]{\mathbb{R}^{#1}}
\newcommand{\bx}{\textbf{x}}
\newcommand{\by}{\textbf{y}}
\newcommand{\bz}{\textbf{z}}
\newcommand{\bu}{\textbf{u}}
\newcommand{\bv}{\textbf{v}}
\newcommand{\I}{\mathcal{I}}


%------------------------------------------------

\usepackage{booktabs} % Required for better table rules
\usepackage{slashbox}
\usepackage{ulem,centernot}


%----------------------------------------------------------------------------------------
%	 TITLE SLIDE
%----------------------------------------------------------------------------------------

\title{Cooperative game theory}

\author{Richard Mužík}

%\titlegraphic{\includegraphics[scale=1.25]{Images/focuslogo.pdf}} % Optional title page image, comment this line to remove it

\institute{richard@imuzik.cz}


%------------------------------------------------

\begin{document}

%------------------------------------------------

\begin{frame}
	\maketitle % Automatically created using the information in the commands above
\end{frame}


\begin{comment}
    - Představit kooperativní hru
    - Cíl: Rozdělení zisku v(N)
    - payoff vektor, imputace
    - jádro (důraz na koaliční racionalitu, tj. vysvětlit, proč je jádro stabilní solution concept)
    - Shapleyho hodnota (pomocí formule i axiomů)
    - třídy her (monotonní, superadditivní, konvexní) a jejich vztahy
\end{comment}

%----------------------------------------------------------------------------------------
%	 SECTION 1
%----------------------------------------------------------------------------------------

\section{Motivations and introduction} % Section title slide, unnumbered

%------------------------------------------------

\begin{frame}{Game theory}
Goal: \textit{To analyse and solve conflicting interaction between subjects}\vspace{0.5in}\\
\pause
Basic model: game in \textbf{normal form} $(N,\mathcal{S},v)$:
\pause
    \begin{itemize}
        \item $N$ ... set of $n$ players
        \pause
        \item $\mathcal{S} = S_1 \times S_2 \times \dots \times S_n$ ... strategy profiles
        \pause
        \begin{itemize}
            \item $S_i$ ... strategies of player $i$
        \end{itemize}
        \pause
        \item $v\colon \mathcal{S} \to \mathbb{R}^n$ ... utility function
        \begin{itemize}
        \pause
            \item $v(S)_i$ ... utility of player $i$ under strategy $S$
        \end{itemize} 
    \end{itemize}
\end{frame}

%------------------------------------------------

%------------------------------------------------

\begin{frame}{Game theory}
Goal: \textit{To analyse and solve conflicting interaction between subjects}\vspace{0.5in}\\
Basic model: game in \textbf{normal form} $(N,\mathcal{S},v)$:
\begin{itemize}
	\item $N$ ... set of $n$ players
	\item $\mathcal{S} = S_1 \times S_2 \times \dots \times S_n$ ... strategy profiles
	\begin{itemize}
		\item $S_i$ ... strategies of player $i$
	\end{itemize}
	\item $v\colon \mathcal{S} \to \mathbb{R}^n$ ... utility function
	\begin{itemize}
		\item $v(S)_i$ ... utility of player $i$ under strategy $S$
	\end{itemize} 
\end{itemize}
Fundamental question: \textit{What strategy should player $i$ choose?}
\end{frame}

%------------------------------------------------

%------------------------------------------------

\begin{frame}{Solution of games in normal form?\\ Nash equilibrium}
	Idea: Deviation from the \textcolor{blue}{actual} strategy to a \red{new} strategy does not improve the outcome.\vspace{0.3in}\\
	\begin{block}{Nash equilibrium}
		Strategy profile $(s_1,\dots,s_n)$ is \textbf{Nash equilibrium}, if it holds for every player $i$,
		\[
			v_i(s_1,\dots,s_{i-1},\textcolor{blue}{s_i},s_{i+1},\dots,s_n) \geq v_i(s_1,\dots,s_{i-1},\textcolor{red}{t_i},s_{i+1},\dots,s_n)
		\]
		for every $\textcolor{red}{t_i} \in S_i$.
	\end{block}
	\begin{itemize}
		\item<2-> Is this model suitable for the analysis of \textit{cooperation}?
	\end{itemize}
\end{frame}

%------------------------------------------------

%------------------------------------------------

\begin{frame}{Prisoner dilemma}
	\begin{tabular}{|l||l|l|}\hline
		\backslashbox{Prisoner A}{Prisoner B}
		&\makebox[3em]{Silence}&\makebox[3em]{Speak}\\\hline
		Silence & -1/-1&-3/0\\\hline
		Speak & 0/-3&-2/-2\\\hline
	\end{tabular}
	\begin{itemize}
		\item<2-> Nash equilibrium does not lead to cooperation
		\item<3-> Main problem?
		\begin{itemize}
			\item<4-> \textbf{The prisoners cannot agree on a strategy in advance}
		\end{itemize}
		\item<5-> Solution?
		\begin{enumerate}
			\item<6-> Fix the existing model 
			\begin{itemize}
			    \item<7-> $\implies$ Pareto optimum
				\item<7-> $\implies$ Nash "cooperative" solution
			\end{itemize}
			\item<9-> Create a new model
			\begin{itemize}
				\item<10-> model of \textit{cooperative games}
			\end{itemize}
		\end{enumerate}
	\end{itemize}
\end{frame}

%------------------------------------------------

%------------------------------------------------

\begin{frame}{Cooperation - examples of models: \textit{Market games}}
	Goal of model? \textit{Understanding of the behaviour of markets}
	\begin{block}{Market}
	   A \textbf{market} $(N,\mathbb{R}^m_+,A,w)$ is given by:
	\begin{itemize}
		\item<2-> $N$ ... set of players
		\item<3-> $\mathbb{R}^m_+$ ... comodity space
		\item<4-> $A \in \mathbb{R}^{m \times n}$ ... matrix of initial allocations
		\begin{itemize}
		    \item<5-> $A = \begin{pmatrix} \mid & \mid & & \mid \\ a^1 & a^2 & \dots & a^n \\ \mid & \mid &  & \mid \\\end{pmatrix}$
			\item<6-> $a^i \in \mathbb{R}^m$ ... commodities owned by player $i$
		\end{itemize}
			\item<7-> $w = w_1 \times w_2 \times \dots w_n$ ... utility function
		\begin{itemize}
			\item<8-> $w_i \colon \mathbb{R}^m_+ \to \mathbb{R}$ ... utility function of player $i$
			\item<9-> $w_i$ is continuous, concave function
		\end{itemize}
	\end{itemize}
	\end{block}
\end{frame}

%------------------------------------------------

%------------------------------------------------

\begin{frame}{Cooperation - examples of models: \textit{Market games}}
	Market games are \textbf{transferable utility} games
	\begin{itemize}
		\item<2-> better understanding after we define cooperative games
		\item<2-> meanwhile: \textit{money} used for evaluation of profit
		\item<3-> formally: $W_i(x,\zeta) = w_i(x) + \zeta$
		\begin{itemize}
			\item<4-> $x \in \mathbb{R}^m$ ... vector of commodities
			\item<4-> $\zeta \in \mathbb{R}$ ... money
			\begin{itemize}
                \item<5-> $\zeta$ may be negative
				\item<5-> amount we paid to buy the commodities
			\end{itemize}
		\end{itemize}
	\end{itemize}
\end{frame}

%------------------------------------------------

%------------------------------------------------

\begin{frame}{Cooperation - examples of models: \textit{Market games}}
	Goal of market games? Determine \textit{trade} on the market
	\begin{block}{Trade of coalition $S$}
		\textbf{Trade} for market $(N,\mathbb{R}^n_+,A,w)$ between players in $\emptyset \neq S \subseteq N$ is a collection $(x^i,\zeta^i)_{i \in S}$ satisfying:
		\begin{enumerate}
                \pause
			\item $\sum_{i \in S}x^i = \sum_{i \in S}a^i$ (preservation of commodities)
                \pause
			\item $\sum_{i \in S}\zeta^i = 0$ (preservation of money).
		\end{enumerate}
	\end{block}
        \pause
	Questions:
        \pause
	\begin{itemize}
            \pause
		\item Which coalitions agree on trading?
		\pause
            \item What trade will occur within a given coalition? 
		\begin{itemize}
			\item \textit{What is the profit of player \textbf{i} for a given market?}
		\end{itemize}
	\end{itemize}
	
\end{frame}

%------------------------------------------------

%------------------------------------------------

\begin{frame}{Cooperation - examples of models: \textit{Voting games}}
    Goal: \textit{Understanding of voting systems a coalition formation}
    \pause
    \begin{block}{Voting game}
        A \textbf{voting game} $(N,\mathcal{W})$ is given by:
    \begin{itemize}
        \pause
        \item $N$ ... set of player
        \pause	
         \item $\mathcal{W} \subseteq 2^N$ ... množina vítězných koalic
        \begin{itemize}
            \pause
            \item $\emptyset \notin \mathcal{W}$
            \item $S \subseteq T \wedge S \in \mathcal{W} \implies T \in \mathcal{W}$
        \end{itemize}
    \end{itemize}
    \end{block}
    \pause
    Special case: \textbf{(Weighted) majority games}
    \begin{itemize}
        \pause
        \item $v_i \in \mathbb{R}$ ... power of player $i$
        \pause
        \item $v(S) = \sum_{i \in S}v_i$ ... power of coalition $S \subseteq N$
        \pause
        \item $S \in \mathcal{W} \iff v(S) \geq k$
        \begin{itemize}
            \item $k \in \mathbb{R}$ ... voting quota
        \end{itemize} 
    \end{itemize}
\end{frame}

%------------------------------------------------

%------------------------------------------------

\begin{frame}{Cooperation - examples of models: \textit{Voting games}}
    Special case: \textbf{(Weighted) majority games}
    \begin{itemize}
        \pause
        \item $v_i \in \mathbb{R}$ ... power of player $i$
        \pause
        \item $v(S) = \sum_{i \in S}v_i$ ... power of coalition $S \subseteq N$
        \pause
        \item $S \in \mathcal{W} \iff v(S) \geq k$
        \begin{itemize}
            \item $k \in \mathbb{R}$ ... voting quota
        \end{itemize} 
    \end{itemize}
    \pause
    Application: Voting system UN Security Council
    \pause
    \begin{itemize}
        \item 15 members (5 permanent)
        \pause
        \begin{itemize}
            \item permanent members: right of veto
        \end{itemize}
        \pause
        \item to adopt resolution: 9 votes needed
    \end{itemize}
    \begin{itemize}
        \pause
        \item $N = \{1,\dots,15\}$
        \pause
        \item $k=39$
        \pause
        \item $v_i = \begin{cases}
            7 & i=1,\dots,5\\
            1 & i=6,\dots,15\\
            \end{cases}$
    \end{itemize}
\end{frame}


%------------------------------------------------

%------------------------------------------------

\begin{frame}{Cooperation - examples of models:\\ \textit{Cost-sharing problem}}
    Goal: \textit{Does cooperation lead to reduction of costs?}
    \begin{itemize}
        \item<2-> $N$ ... set of players
        \item<3-> $c(S)$ ... value of costs for a service shared by $S \subseteq N$
        \begin{itemize}
            \item<4-> $c(S) + c(T) \geq c(S \cup T)$
            \item<5-> \small{a common solution simulated by union of individual solutions}
        \end{itemize}
    \end{itemize}
    Examples:
    \begin{itemize}
        \item<6-> drinking water supply
        \item<7-> municipal waste collection
        \item<8-> subscriptions to scientific journals
    \end{itemize}
	
\end{frame}


%------------------------------------------------

%------------------------------------------------

\begin{frame}{Cooperation - examples of models:\\ \textit{Minimal spanning-tree games}}
	Goal: \textit{Find the best connection of players to a source}
	\begin{itemize}
		\item<2-> $N = N' \cup \{0\}$ ... set of players + source
		\item<3-> $c_{ij}$ ... cost of connecting $i,j$
		\item<4-> solution: a network, where each $i \in N$ is connected to $0$ with minimal sum of costs
	\end{itemize}
	
\end{frame}


%------------------------------------------------

%------------------------------------------------

\begin{frame}{Cooperative game}
    \begin{block}{Cooperative game}
        \pause
        A \textbf{cooperative game} is an ordered pair $(N,v)$, where $N$ is a set of players and $v\colon 2^N \to \mathbb{R}$ is the characteristic function. Further, $v(\emptyset) = 0$.
    \end{block}
    \begin{itemize}
        \item<5-> $\Gamma^n$ ... set of $n$-person cooperative games
        \item<3-> $S \subseteq N$ ... coalition
        \item<4-> $v(S)$ ... values of coalition
        \item<6-> usually $N = \{1,\dots,n\}$
        \begin{itemize}
            \item<7-> $(S,v_S)$ is \textbf{subgame} $(N,v)$:
            \begin{itemize}
                \item<7-> $v_S \colon 2^S \to \mathbb{R}$
                \item<7-> $v_S(T) \coloneqq v(T)$ pro $T \subseteq S$
            \end{itemize}
        \end{itemize}
    \end{itemize}
	
\end{frame}

%------------------------------------------------

%------------------------------------------------

\begin{frame}{Cooperation - examples of models: \textit{Market games}}
 	\begin{itemize}
 		\item $(N,\mathbb{R}^m_+,A,w)$ ... market
	\end{itemize}
	\begin{block}{Feasible S-allocation}
		\textbf{Feasible $S$-allocations} is $(a_S^i)_{i \in S}$ satisfying
		\[
		\sum_{i \in S}a^i_S = \sum_{i \in S}a^i.
		\]
		We denote the set of feasible $S$-allocations by $\mathcal{A}_S$.
	\end{block}
\end{frame}

%------------------------------------------------

%------------------------------------------------

\begin{frame}{Cooperation - examples of models: \textit{Market games}}
	\begin{itemize}
		\item $(N,\mathbb{R}^m_+,a,w)$ ... market
	\end{itemize}
	\begin{block}{Feasible S-allocation}
		\textbf{Feasible S-allocation} is $(a_S^i)_{i \in S}$ satisfying $\sum_{i \in S}a^i_S = \sum_{i \in S}a^i$.\\
		We denote the set of feasible $S$-allocations by $\mathcal{A}_S$.
	\end{block}
	\begin{block}{Market game}
		Cooperative game $(N,v)$ is \textbf{market game}, if there is market $(N,\mathbb{R}^m_+,A,w)$ satisfying
		\[
		v(S) = \max\{\sum_{i \in S}w^i(a_S^i)\mid (a_S^i)_{i \in S} \in \mathcal{A}_S\}.
		\]
	\end{block}
\end{frame}

%------------------------------------------------

%------------------------------------------------

\begin{frame}{Cooperation - examples of models: \textit{Voting games}}
    \begin{block}{Voting game}
        A \textbf{voting game} $(N,\mathcal{W})$ is given by:
        \begin{itemize}
            \item $N$ ... set of players
            \item $\mathcal{W} \subseteq 2^N$ ... set of winning coalitions
            \begin{itemize}
                \item $\emptyset \notin \mathcal{W}$
                \item $S \subseteq T \wedge S \in \mathcal{W} \implies T \in \mathcal{W}$
            \end{itemize}
        \end{itemize}
    \end{block}
    \pause
    \begin{block}{Simple game}
        \textbf{Simple game} $(N,v)$ satisfies:
        \begin{enumerate}
            \item $v\colon 2^N \to \{0,1\}$
            \item $S \subseteq T \wedge v(S)=1 \implies v(T)=1$
        \end{enumerate}
    \end{block}
    \begin{itemize}
        \pause
        \item $v(S)=1 \iff S \in \mathcal{W}$
    \end{itemize}
 
\end{frame}

%------------------------------------------------

%------------------------------------------------

\begin{frame}{Cooperation - examples of models:\\ \textit{Cost-sharing problem}}
    Goal: \textit{Does cooperation lead to reduction of costs?}
    \begin{itemize}
        \item $N$ ... set of players
        \item $c(S)$ ... value of costs for a service shared by $S \subseteq N$
        \begin{itemize}
            \item $c(S) + c(T) \geq c(S \cup T)$
            \item \small{a common solution simulated by union of individual solutions}
        \end{itemize}
    \end{itemize}
    \pause
    \textbf{Cooperative game}
    \begin{itemize}
        \item<3-> $c\colon 2 ^N \to \mathbb{R}$
        \item<3-> a so called \textit{cost game} $(N,c)$
        \begin{itemize}
            \item<4-> $c(S)$ ... values represent \textit{costs}
            \item<4-> $v(S)$ ... values represent \textit{profit}
        \end{itemize}
    \end{itemize}
\end{frame}

%------------------------------------------------

%DALSI PRIKLADY SAMI?



%------------------------------------------------

\begin{frame}{Classes of games}
	Example: \textit{Bigger coalitions are better coalitions...}
	\begin{itemize}
	    \item<2-> \textbf{monotonic game} ($S \subseteq T \subseteq N$) \[v(S) \leq v(T)\]
	    \item<3-> \textbf{superadditive game} ($S,T \subseteq N, S \cap T = \emptyset$) \[v(S) + v(T) \leq v(S \cup T)\]
	    \vspace{-0.3cm}
	    \begin{itemize}
			\item<4-> $S\cup T$ can behave as separate $S$ and $T$
			\item<5-> does not hold if there are costs of managing coalitions
			\item<6-> cost-games: $c(S) + c(T) \geq v(S \cup T)$ ... \textbf{subadditive game}
		\end{itemize}
	    \item<7-> \textbf{convex game} ($S,T \subseteq N$) \[v(S) + v(T) \leq \textcolor{red}{v(S \cap T)} +  v(S \cup T)\]
	\end{itemize}
\end{frame}

%------------------------------------------------

%------------------------------------------------

\begin{frame}{Classes of games}
    Example: \textit{Bigger coalitions are better coalitions...}
    \begin{itemize}
        \item \textbf{monotonic game} ($S \subseteq T \subseteq N$) \[v(S) \leq v(T)\]
        \item \textbf{superadditive game} ($S,T \subseteq N, S \cap T = \emptyset$) \[v(S) + v(T) \leq v(S \cup T)\]
        \item \textbf{convex game} ($S,T \subseteq N$) \[v(S) + v(T) \leq \textcolor{red}{v(S \cap T)} +  v(S \cup T)\]
    \begin{itemize}
        \item<2-> also \textbf{supermodular game}
        \item<3-> cost-game: $c(S) + c(T) \geq c(S \cap T) + c(S \cup T)$ 
        \begin{itemize}
            \item<3-> \textbf{concave (submodular)}
        \end{itemize}
    \end{itemize}
\end{itemize}
\end{frame}

%------------------------------------------------

%------------------------------------------------

\begin{frame}{Goal of the model of cooperative games}
	\textit{Money first!}
	\vspace{0.3in}
	\begin{itemize}
	    \item<2-> \textbf{Payoff vector} $\bx \in \Rn$
	    \begin{itemize}
	        \item<2-> $x_i$ represents payoff of player $i$
	    \end{itemize}
	    \item<3-> Vector $\bx \in \Rn$ is \textbf{efficient}, if $\sum_{i \in N}x_i = v(N)$
	    \begin{itemize}
	        \item<4-> Usually, we distribute $v(N)$
            \begin{enumerate}
                \item<4-> \textit{value of cooperation} $v(N)$
                \item<4-> \textit{shared costs} $c(N)$
            \end{enumerate}
	    \end{itemize}
	    \item<5-> Vector $\bx \in \Rn$ is \textbf{individually rational}, if $x_i \geq v(i)$
	    \begin{itemize}
	        \item<6-> players prefer $x_i$ over $v(i)$
	    \end{itemize}
	    \end{itemize}
	    \vspace{0.2in}
	    \begin{itemize}
	    \item<7-> $\I^*(v) = \left\{x \in \Rn \mid x(N) = v(N) \right\}$ ... \textbf{preimputation}
	    \begin{itemize}
	        \item<7-> $x(S)\coloneqq \sum_{i \in S} x_i$
	    \end{itemize}
	    \item<8-> $\I(v) = \left\{x \in \I^*(v) \mid \forall i \in N: x_i \geq v(i)\right\}$ ... \textbf{imputation}
	\end{itemize}
\end{frame}

%------------------------------------------------


%------------------------------------------------

\begin{frame}{Solution concepts}
	\begin{itemize}
	    \item<2-> Set of payoff vectors satisfying further properties are \textbf{solution concepts}
	    \item<3-> Can reflect payoff distribution, which is
	    \begin{itemize}
	        \item<4-> \textit{...fair...}
	        \item<4-> \textit{...non-discriminatory...}
	        \item<4-> \textit{...stable (players will accept it)...} 
	        \item<4-> \textit{...}
	    \end{itemize}
	\end{itemize}
\end{frame}

%------------------------------------------------

%------------------------------------------------

\begin{frame}{Examples of solution concepts: The core}
    Idea: \textit{Payoff distribution leads to cooperation...}
    \begin{block}{The core}
    For a cooperative game $(N,v)$, the \textbf{core} $\mathcal{C}(v)$ is
    \[\mathcal{C}(v) = \left\{x \in \I^*(v) \mid x(S) \geq v(S), \forall S \subseteq N \right\}.\]
    \end{block}
    \begin{itemize}
        \item<2-> assumption: \textit{homo economicus} 
        \begin{itemize}
            \item<3-> model of human as a player
            \item<3-> strictly rational and selfish
            \item<3-> follows his subjective goals
        \end{itemize}
        \item<4-> $v(N)$ ... value, which is distributed among players
            \item<5-> $x(S) > v(S) \implies$ coalition $S$ does not leave $N$ 
            \begin{itemize}
                \item<6-> would lead to $(S,v_S)$
                \item<6-> $v(S)$ ... distributed value
            \end{itemize}
    \end{itemize}
\end{frame}

%------------------------------------------------

%------------------------------------------------

\begin{frame}{Examples of solution concepts: The Shapley value}
    Idea: \textit{Divide the profit in a fair way...}
    \begin{block}{The Shapley value}
        For a cooperative game $(N,v)$, the \textbf{Shapley value} $\phi(v)$ of player $i$ is
        \[
        \phi_i(v) = \sum_{S \subseteq N \setminus i}\frac{s!(n-s-1)!}{n!}\left(v(S \cup i) - v(S)\right)
        \]
    \end{block}
    \begin{itemize}
        \item<2-> $v(S\cup i)-v(S)$
        \begin{itemize}
            \item<3-> \textit{marginal contribution} (of player $i$ in $S \cup i$)
        \end{itemize}
        \item<2-> $\frac{s!(n-s-1)!}{n!}$ 
        \begin{itemize}
            \item<4-> weights reflecting different sizes of coalitions
        \end{itemize}
        \item<2-> $\sum_{S \subseteq N \setminus i}$
        \begin{itemize}
            \item<5-> sum of all marginal contributions of $i$
        \end{itemize}
    \end{itemize}
\end{frame}

%------------------------------------------------

%------------------------------------------------

\begin{frame}{Solution concepts}
    Formally:
	\begin{enumerate}
	    \item<2-> sets of payoff vectors 
	    \begin{itemize}
	        \item<3-> $\Sigma(v) = \left\{x \in \Rn \mid \dots\right\}$
	    \end{itemize}
	    \item<2-> functions on games
	    \begin{itemize}
	        \item<4-> $\Sigma \colon \Gamma^n \to 2^{\Rn}$
	    \end{itemize}
	\end{enumerate}
\end{frame}

%------------------------------------------------

%------------------------------------------------

\begin{frame}{Solution concepts}
    Formally:
	\begin{enumerate}
	    \item sets of payoff vectors
	    \begin{itemize}
	        \item $\Sigma(v) = \left\{x \in \Rn \mid \dots\right\}$
	    \end{itemize}
	    \item functions on games
	    \begin{itemize}
	        \item $\Sigma \colon \Gamma^n \to 2^{\Rn}$
	    \end{itemize}
	\end{enumerate}
	We distringush
	\begin{enumerate}
	    \item<2-> \textbf{single-point} solution concepts
	    \begin{itemize}
	        \item<3-> as a set: $\Sigma(v) = \{x\}$ 
	        \begin{itemize}
	            \item<3-> we prefer: $\Sigma(v) = x$
	        \end{itemize}
	        \item<4-> as a function: $\Sigma \colon \Gamma^n \to \mathbb{R}$
	    \end{itemize}
	    \item<2-> \textbf{multi-point} solution concepts
	\end{enumerate}
\end{frame}

%------------------------------------------------

%------------------------------------------------

\begin{frame}{Alternative way to define the Shapley value}
    \textit{It is possible to define it using its properties...}
    \begin{block}{The Shapley value}
       The \textbf{Shapley value} $\phi(v)$ is the only function $f \colon \Gamma^n \to \mathbb{R}$ satisfying for all games $(N,v),(N,w)$:
        \begin{enumerate}
            \item (AXIOM OF EFFICIENCE)
            \begin{itemize}
                \item $\sum_{i \in N}f_i(v) = v(N)$
            \end{itemize}
            \item (AXIOM OF SYMMETRY)
            \begin{itemize}
                \item $\forall i,j \in N$ $(\forall S \subseteq N \setminus \{i,j\}: v(S \cup i) = v(S \cup j)) \implies f_i(v) = f_j(v)$
            \end{itemize}
            \item (AXIOM OF NULL PLAYER)
            \begin{itemize}
                \item $\forall i \in N$ $(\forall S \subseteq N: v(S) = v(S \cup i)) \implies f_i(v)=0$
            \end{itemize}
            \item (AXIOM OF ADDITIVITY)
            \begin{itemize}
                \item $v,w \in \Gamma^n: f(v+w)=f(v)+f(w)$
            \end{itemize}
        \end{enumerate}
        
    \end{block}
\end{frame}

%------------------------------------------------

%------------------------------------------------

\begin{frame}{Summary}
    \begin{alertblock}{Cooperative games}
        A \textbf{cooperative game} is given by a set of players and real values representing the profit of each subset of players (\textit{coalition}). These values are encoded by the \textit{characteristic function} of a game. The goal of the cooperative game theory is to find payoffs (in form of \textbf{payoff vectors}) for individual players based on values of the game. Payoff vectors satisfying further properties form \textbf{solution concepts}. These can be expressed as:
        \begin{enumerate}
            \item sets of payoff vectors
            \item functions on games
            \begin{enumerate}
                \item defined by formula
                \item defined by its properties
            \end{enumerate}
        \end{enumerate}
    \end{alertblock}
\end{frame}

%------------------------------------------------

	


\end{document}
